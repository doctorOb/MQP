\documentclass[a4wide,10pt]{extarticle}
\usepackage[margin=0.5in]{geometry}
\usepackage{enumitem}
\setlist{noitemsep}
\usepackage{color}
\usepackage{array}
\usepackage{hyperref}
\usepackage[compact]{titlesec}
\titlespacing*{\section}{0pt}{6pt}{6pt}
\titlespacing*{\subsection}{0pt}{6pt}{6pt}

\begin{document}
\thispagestyle{empty}

\begin{center}
\textbf{MQP Meeting minutes}
\vspace{0.33cm}
\end{center}

\begin{center}
\begin{tabular}{| m{2.8cm} | m{13.6cm} |} \hline
\textbf{Date and Time} & Tuesday 26, November 2013 at 2:00 pm \\ \hline
\textbf{Venue} & Craig's Office \\ \hline
\textbf{Participants} & Curtis, Craig, Krishna, Dan\\ \hline
\end{tabular}
\end{center}

\vspace{0.5cm}
\begin{center}
\begin{tabular}{| m{3.0cm} | m{12.6cm} | m{2cm}|} \hline
\textbf{Item} & \textbf{Notes and Discussion}\\ \hline

Notes on the Demo & 
	\begin{itemize}
		\item Could this be extended to dispatch entire HTTP requests to peer routers to speed up load times for typical web pages (Krishna).
		\item Try to devise a more eloquent way to determine when to aggregate (perhaps a conditional GET on content length)
		\item Monitor performance using tshark or tcpdump. (Throughput, )
	\end{itemize} 
\\ \hline

Trust Calc (Pt. 3) &
	\begin{itemize}
		\item Running trust calculation with a bit vector. Penalize recent activity more then past actions. 
		\item Good action adds a 1 to the front of the vector, bad actions add a 0. Divide by $2^{number-of-actions}$. Produces a trust value.
		\item Highly reactionary. 1 sample (leading bit) will always determine 50\% of the value.
		\item Might work better for averaging past transactions, or for computing reliability.
		\item What does trust mean? Accuracy, Timeliness (next to godliness), willingness, reliability. We're better of having a Pure Quality metric which lumps all of these metrics together.
	\end{itemize}
\\ \hline

Determining when a response is false &
	\begin{itemize}
		\item End of the day, the response is either good data or bad data. How can we truly know? Zero Knowledge proof only works on static content.
		\item HTTP headers (last modified date) might allow us to conclude when the server has responded with cached or out of date data.
		\item A server is more likely to mess up with a response then the peer. Verifying data is a very difficult problem.
		\item There will be many cases in which we can't verify the data. The heavier the analysis, the more overhead is introduced.
		\item An accusation of lying will require a lot of work (Craig)
		\item This was scoped more towards static content like movie files, we shouldn't have to worry to much about dealing with dynamic content.
	\end{itemize}
\\ \hline

For next week &
	\begin{itemize}
		\item Excel timeline for upcoming work.
		\item Revision of paper for Craig on Sunday (read it!)
	\end{itemize}
\\ \hline


\end{tabular}
\end{center}
\end{document}