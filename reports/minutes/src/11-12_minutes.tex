\documentclass[a4wide,10pt]{extarticle}
\usepackage[margin=0.5in]{geometry}
\usepackage{enumitem}
\setlist{noitemsep}
\usepackage{color}
\usepackage{array}
\usepackage{hyperref}
\usepackage[compact]{titlesec}
\titlespacing*{\section}{0pt}{6pt}{6pt}
\titlespacing*{\subsection}{0pt}{6pt}{6pt}

\begin{document}
\thispagestyle{empty}

\begin{center}
\textbf{MQP Meeting minutes}
\vspace{0.33cm}
\end{center}

\begin{center}
\begin{tabular}{| m{2.8cm} | m{13.6cm} |} \hline
\textbf{Date and Time} & Tuesday 12 November 2013 at 2:00 pm \\ \hline
\textbf{Venue} & Craig's Office \\ \hline
\textbf{Participants} & Curtis, Craig, Dan\\ \hline
\end{tabular}
\end{center}

\vspace{0.5cm}
\begin{center}
\begin{tabular}{| m{3.0cm} | m{12.6cm} | m{2cm}|} \hline
\textbf{Item} & \textbf{Notes and Discussion}\\ \hline

Using HTTP for intra router communication & 
	\begin{itemize}
		\item Provides a consistent header for keeping track of control and session information.
		\item Twisted already provides a lot of helper functionality for HTTP.
		\item More of a convenience gain. Adding logic to the control information and header passing would be as simple as including an extra field in the HTTP header. Saves on development time vs writing a custom header and parser from scratch.
		\item Why not use a static struct? Ease of development. 
		\item Make a diagram of the protocol, pen and paper is fine for now.
	\end{itemize} 
\\ \hline

Trust Model &
	\begin{itemize}
		\item Implement after the prototype is flushed out and finalized. Good for discussion section.
		\item Write about it like we're actually implementing it.
		\item Reputation based or accumulative (local) trust platform?
		\item In our scenario, we'll likely be dealing with less then 10 neighbors for a long period of time, so the reputation model popularized by P2P networks is not necessary.
		\item Borrow trust calculating equations from P2P networks.
	\end{itemize}
\\ \hline

Liability with asymetric cryptography &
	\begin{itemize}
		\item Sender produces a digital signature of a hashed portion of the initial session request.
		\item associate each request with IP, time, url, port, signature (to prove identity).
		\item Each router holds public key of every neighbor (tied to IP) to use for verifying signature
		\item Working trust value for each peer (weighted average). Alpha (recent activity), Beta (past activity) paramaters, design protocol to weigh them differently. We should probably weigh Alpha higher then Beta.
	\end{itemize}
\\ \hline

HTTPS support &
	\begin{itemize}
		\item Can we support HTTPS easily? Is it worthwhile?
		\item Recent news: HTTP 2.0 will use HTTPS by default. ``C. HTTP/2 to only be used with https:// URIs on the open Internet. http:// URIs would continue to use HTTP/1 (and of course it would
still be possible for older HTTP/1 clients to still interoperate with https:// URIs).''
		\item Can we mimick one HTTPS session across multiple routers (instead of using HTTPS on each router).
		\item At least talk about the idea and the trade offs.
	\end{itemize}
\\ \hline

Metrics for choosing Peers &
	\begin{itemize}
		\item Speed vs security
		\item trust and reliability metric
	\end{itemize}
\\ \hline

\end{tabular}
\end{center}
\end{document}