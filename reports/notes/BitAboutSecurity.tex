\documentclass{article}
\begin{document}

The goal of this project is to establish a bandwidth aggregation network between mutually connected Internet routers, such that users may leverage neighboring idle networks to split large file downloads up. In this scenario, a client will use any number of his neighbors IP addresses in order to download any content. This opens up all sorts of legal and security concerns. 

\subsection{Liability for Open Wifi network Owners}

There are no laws on the books that explicitly state that it is illegal to operate an open wifi network. But, people with such networks can and have been prosecuted for the content anonymous users downloaded through their networks. The argument that has the most leway is negligence. Copyright lawyers often use the negligence claim in suits against open network owners who unknowingly facilitated the distribution of protected content.

\begin{quotation}``When you do something careless, and that carelessness costs someone else money, you pay the carelessness tax"--Mark Randazza\end{quotation}

The core elements of the negligence claim can be broken down to apply to open networks.
\begin{itemize}
\item Duty -- did the defendant have a duty? (to keep your wifi secured)
\item Breach -- did the defendant breach that duty (facilitated an open or semi--open network)
\item Causation -- was the breach the cause of the plaintif\'s damages?(did their network lead to copyright infringement)
\item Damages -- were there damages, if so, how much? (copyright infringement, illegal Child Pornography, pressure cooker queries)
\end{itemize}

The open--wifi defense gives the defendant plausible deniability, but a negligence claim can still be filed. However, it's important to remember that the use of this claim in copyright scenarios in particular is still iffy. As no higher precedent has been set, it is up to the jury presiding over each case to determine guilt. However, it is a legality concern that most certianly needs to be addressed by this MQP on some level.

While this MQP does not require neighbor networks to be open, they must be configured to allows cooperating routers to comunicate with them, which is where the negligence claim can still apply. Thus, peers would have to mutually aggree to the liability they have for not only their browsing, but for their neighbors as well. 

\end{document}